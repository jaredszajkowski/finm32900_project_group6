% !TeX root = project.tex
\newcommand*{\PathToAssets}{../assets}%
\newcommand*{\PathToOutput}{../_output}%

%%%%%%%%%%%%%%%%%%%%%%%%%%%%%%%%%%%%%%
%% This file is compiled with XeLaTex.
%%%%%%%%%%%%%%%%%%%%%%%%%%%%%%%%%%%%%%
\documentclass[12pt]{article}
\usepackage{my_article_header}
\usepackage{my_common_header}
\usepackage{graphicx}

\begin{document}
\title{
Market Expectations in the Cross-Section of Present Values
}

\author{
Ilya Melnikov, Jared Szajkowski, Zac Johnson\
}
\begin{titlepage}
\maketitle

\doublespacing
\begin{abstract}
    From the abstract of "Market Expectations in the Cross-Section of Present Values", Kelly and Pruitt (2013):

    Returns and cash flow growth for the aggregate U.S. stock market are highly and robustly predictable. Using 
    a single factor extracted from the cross-section of book-to-market ratios, we find an out-of-sample return 
    forecasting R\textsuperscript{2} of 13\% at the annual frequency (0.9\% monthly). We document similar out-of-sample 
    predictability for returns on value, size, momentum, and industry portfolios. We present a model linking 
    aggregate market expectations to disaggregated valuation ratios in a latent factor system. Spreads in value 
    portfolios’ exposures to economic shocks are key to identifying predictability and are consistent with 
    duration-based theories of the value premium.

\end{abstract}

\end{titlepage}

\doublespacing
\section{Out-Of-Sample Forecasted Values vs Actual Values}
Here we present the plots with the out-of-sample forecasted values vs actual values for the 6, 25, and 100 portfolios.
\begin{figure}[h]
    \centering
    \includegraphics[width=0.8\textwidth]{plots/Monthly_Out_of_Sample_Forecasts_for_6_Portfolios_Portfolio_Data.png}
    \caption{Monthly Out-of-Sample Forecast for 6-Portfolios Portfolio Data}
    \label{fig:forecast_6_monthly}
\end{figure}

\begin{figure}[h]
    \centering
    \includegraphics[width=0.8\textwidth]{plots/Annual_Out_of_Sample_Forecasts_for_6_Portfolios_Portfolio_Data.png}
    \caption{Annual Out-of-Sample Forecast for 6-Portfolios Portfolio Data}
    \label{fig:forecast_6_annual}
\end{figure}

\begin{figure}[h]
    \centering
    \includegraphics[width=0.8\textwidth]{plots/Monthly_Out_of_Sample_Forecasts_for_25_Portfolios_Portfolio_Data.png}
    \caption{Monthly Out-of-Sample Forecast for 25-Portfolios Portfolio Data}
    \label{fig:forecast_25_monthly}
\end{figure}

\begin{figure}[h]
    \centering
    \includegraphics[width=0.8\textwidth]{plots/Annual_Out_of_Sample_Forecasts_for_25_Portfolios_Portfolio_Data.png}
    \caption{Annual Out-of-Sample Forecast for 25-Portfolios Portfolio Data}
    \label{fig:forecast_25_annual}
\end{figure}

\begin{figure}[h]
    \centering
    \includegraphics[width=0.8\textwidth]{plots/Monthly_Out_of_Sample_Forecasts_for_100_Portfolios_Portfolio_Data.png}
    \caption{Monthly Out-of-Sample Forecast for 100-Portfolios Portfolio Data}
    \label{fig:forecast_100_monthly}
\end{figure}

\begin{figure}[h]
    \centering
    \includegraphics[width=0.8\textwidth]{plots/Annual_Out_of_Sample_Forecasts_for_100_Portfolios_Portfolio_Data.png}
    \caption{Annual Out-of-Sample Forecast for 100-Portfolios Portfolio Data}
    \label{fig:forecast_100_annual}
\end{figure}


\doublespacing
\section{Regression Results}

As follows are the regression results for the 6, 25, and 100 portfolios. The tables include the in-sample and 
out-of-sample R-squared values for the monthly and annual regressions.

% 6 Portfolios Monthly Regression Results
\begin{table}[ht]
    \centering
    \caption{6 Portfolios Monthly Regression Results}
    \input{tables/summary_table_6_monthly.tex}
\end{table}

% 6 Portfolios Annual Regression Results
\begin{table}[ht]
    \centering
    \caption{6 Portfolios Annual Regression Results}
    \input{tables/summary_table_6_annual.tex}
\end{table}

% 25 Portfolios Monthly Regression Results
\begin{table}[ht]
    \centering
    \caption{25 Portfolios Monthly Regression Results}
    \input{tables/summary_table_25_monthly.tex}
\end{table}

% 25 Portfolios Annual Regression Results
\begin{table}[ht]
    \centering
    \caption{25 Portfolios Annual Regression Results}
    \input{tables/summary_table_25_annual.tex}
\end{table}

% 100 Portfolios Monthly Regression Results
\begin{table}[ht]
    \centering
    \caption{100 Portfolios Monthly Regression Results}
    \input{tables/summary_table_100_monthly.tex}
\end{table}

% 100 Portfolios Annual Regression Results
\begin{table}[ht]
    \centering
    \caption{100 Portfolios Annual Regression Results}
    \input{tables/summary_table_100_annual.tex}
\end{table}


\doublespacing
\section{Replication Overview}

The replication of Table 1 from Kelly and Pruitt (2013) aimed to validate the 
findings on market return predictability using book-to-market ratios. In this project we 
attempted to reconstruct the key univariate predictor by applying partial least squares (PLS) 
regressions to the market returns and book-to-market ratios, with the goal of reproducing 
the original study’s results.

\doublespacing
\section{Successes}
We were successful in accessing, cleaning, and manipulating the required data sets, specifically 
the Fama-French book-to-market portfolios and the CRSP value weighted market return data. 
We were also successful in writing python code that we believe correctly set up PLS regression 
equations and the recursive prediction.

\doublespacing
\section{Challenges}

The initial data acquisition was straight-forward due to the simplicity of the data sources, but the
challenge began when we attempted to align our data sets with those in the academic paper.

From the paper:

"Our annual forecasts consist of overlapping monthly observations
that span 81 years, while our monthly forecasts use 972 nonoverlapping time-
series observations."

And then later:

"Let time indices represent months. Consider a forecast for the return rτ +12
that is realized over the 12-month period t + 1 to t + 12."

This statement implies that the data is shifted by 1 month when running the regressions.

However, there is confusion in the paper about the exact time period of the data shift, based on the
following statement:

"We use book-to-market ratios for Fama and French’s (1993) size- and value-
sorted portfolios (in which U.S. stocks are divided into 6, 25, or 100 portfolios).
Data files from Ken French’s website report monthly portfolio-level market
equity value and annual portfolio book equity value, which we use to construct
a monthly panel of portfolio book-to-market ratios. A book-to-market ratio in
month t uses a portfolio’s total market capitalization at the end of month t and
the latest observable annual book equity total for the portfolio. We assume that
portfolio book equity in calendar year Y becomes observable after June of year
Y + 1 following Fama and French (1993)."

The above statement implies that there is a shift between the end of the fiscal year (when book equity
would be "finalized") and June of the following year (when book equity would be observable), which could 
be as much as 6 months. This is in direct conflict with the statement above that imply that the data is 
shifted by only 1 month.

There is further confusion as to which set of book-to-market ratios is used in the regressions. The Ken French data 
provides 2 sets of ratios; one set is based on the end of the fiscal year (December), and the other is based on the end of June.

As an example, from the 6 portfolios data, the book-to-market ratio options are as follows:


6 : For portfolios formed in June of year t   

Value Weight Average of:

\[
\frac{BE}{ME}
\]

Calculated for June of t to June of t+1 as:    

\[
\frac{\sum \left( ME(Mth) \times \frac{BE(Fiscal\ Year\ t-1)}{ME(Dec\ t-1)} \right)}
     {\sum ME(Mth)}
\]

Where Mth is a month from June of t to June of t+1 and BE(Fiscal Year t-1) is adjusted for net stock issuance to Dec t-1 (1129 rows x 6 cols)


7 : For portfolios formed in June of year t   

Value Weight Average of:

\[
\frac{BE_FYt-1}{ME_June t}
\]

Calculated for June of t to June of t+1 as:    

\[
\frac{\sum \left( ME(Mth) \times \frac{BE(Fiscal\ Year\ t-1)}{ME(Jun\ t)} \right)}
     {\sum ME(Mth)}
\]

Where Mth is a month from June of t to June of t+1 and BE(Fiscal Year t-1) is adjusted for net stock issuance to Jun t (1129 rows x 6 cols)



There was signifiant time spent attempting to align the data sets and the regressions with the academic paper, but 
ultimately we were unable to do so in a way that could reproduce the results from the paper.




\doublespacing
\section{Data Sources}

CRSP: Provided market returns, stock prices, and trading volume data.
Fama-French Factors: Used as a robustness check to compare return predictability against standard factor-based models.

\end{document}